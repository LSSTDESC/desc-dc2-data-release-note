%%%%% MACROS AND SETTINGS FOR CS REPORTS

%=====================================================================
% SET UP RECOMMENDATION COMMANDS
%
% Adapted from Science Roadmap file, Rachel Bean & Phil Marshall Summer 2015
% Based on commands written by Daniel Einsenstein for DESI SRP April 2015
%=====================================================================
\usepackage{xifthen}
\usepackage{xspace}

% Number tables and figures by section
\let\counterwithout\relax
\let\counterwithin\relax
\usepackage{chngcntr}
\counterwithin{table}{section}
\counterwithin{figure}{section}

\usepackage{multirow}
% Enable sensible resizing of lists of projects and recommendations:
\usepackage{tocloft}
\renewcommand\cftparafont{\small}
\renewcommand\cftparapagefont{\small}
\renewcommand\cftsubparafont{\footnotesize}
\renewcommand\cftsubparapagefont{\footnotesize}

% for \autoref
\renewcommand*\sectionautorefname{Section}
\renewcommand*\subsectionautorefname{Section}
\renewcommand*\subsubsectionautorefname{Section}


\usepackage{datetime}

%\newcommand{\docheader}{{LSST DESC Operations Plan}}

% --------------------------------------------------------------------

\makeatletter

% --------------------------------------------------------------------
% REFERENCING COMMANDS

% Include down to the subsection level in the table of contents:
\setcounter{tocdepth}{2}

% \counterwithin{section}{chapter}
% \counterwithin{table}{chapter}
% \counterwithin{figure}{chapter}

\renewcommand*{\chapterautorefname}{Chapter}
\renewcommand*{\appendixautorefname}{Appendix}
\newcommand{\appref}[1]{\hyperref[#1]{Appendix~\ref{#1}}}

\newcommand{\relabel}[3]{\@bsphack
    \protected@write\@auxout{}%
        % {\string\newlabel{#2}{{#1}{#3}{\relax}{#2}{}}}%
        {\string\newlabel{#2}{{#1}{\thepage}{\relax}{#3}{}}}%
    \@esphack}

% --------------------------------------------------------------------

% POINT OF CONTACT

% Contact list - repo in URL must be set correctly to enable queries to be made:
\newcommand{\contact}[2]{%
\href{https://github.com/LSSTDESC/CosmologicalSimulationsPlan/issues/new?title=Question:\%20\&body=#2:\%20}{{\it #1}}%
}
\newcommand{\dontcontact}[2]{{\it #1}}

% Link to issue thread:
\newcommand{\issue}[1]{%
\href{https://github.com/LSSTDESC/CosmologicalSimulationsPlan/issues/#1}{\#{#1}}%
}


% --------------------------------------------------------------------

% USEFUL MACROS

% The \todo macro can be used to mark missing items or to pose questions to the group.
\newcommand{\todo}[1]{\noindent{\textcolor{red}{[TODO: #1]}}}
\newcommand{\toask}[1]{\noindent{\textcolor{magenta}{[Q for #1]}}}
\newcommand{\note}[1]{\noindent{\textcolor{blue}{[#1]}}}


% Uncomment the following line to make the \todo items invisible.
%\renewcommand\todo[1]{\relax}

% List of Recommendations, using Table of Contents commands:
\newcommand{\listofrecommendations}{
  \@starttoc{tor}
}

% For inline code (like `...` in markdown)
\DeclareUrlCommand\code{\urlstyle{tt}}

% List of projects, designed to act as a table of contents in each
% section:
\newcommand{\listofjusttheserecommendations}[1]{
  \vskip\baselineskip
  {\bf \nameref{sec:\headerstring} Recommendations:}
  \phantomsection\label{tab:\headerstring:recommendations}
  \medskip
  \hrule
  % \@starttoc{#1}
  \@starttoc{\headerstring.tor}
  \hspace{-0.5em}
  \medskip
  \hrule
  \vskip\baselineskip
  \vskip\baselineskip
}

% --------------------------------------------------------------------

% COUNTERS AND VARIABLES:

\newcounter{recommendation}

\newcommand{\resetnumbering}{
   \setcounter{subsubsection}{0}
   \setcounter{paragraph}{0}
   \setcounter{subparagraph}{0}
   \setcounter{equation}{0}
   \setcounter{recommendation}{0}
}

% Prefix section title
\newcommand{\sectionstring}{}
% Prefix for key projects and recommendations in each section (SL, WL etc)
\newcommand{\headerstring}{}
% Recommendations are organised by sections, whose names
% are used in formatting the titles etc. ENV, FRW, WKF, OPS

\def\myaddcontentsline#1#2#3#4{\addtocontents{#1}{\protect\contentsline{#2}{#3}{#4}{}}}

% --------------------------------------------------------------------

% TABLE OF CONTENTS

\newcommand{\maketoc}{
   \renewcommand{\contentsname}{\hrule{\large\flushleft\sffamily\bfseries Contents\vspace{-1.2cm}}}
   \setlength{\parskip}{0.1cm}
   \phantomsection\tableofcontents % PJM: phantom section is needed to give \label something to point at.
   \flushbottom
   \vs\hrule
}


% --------------------------------------------------------------------

% RECOMMENDATION COMMANDS

%%% Recommendations should be phrased as verbs
%%% Recommendations have one argument: a title

%%% Use \recommendation[tag]{title}
%%% \recommendationref{tag} will echo the recommendation number
%%% \pageref{tag} will return the recommendation number

% \recommendationnum is used for labeling recommendations. Since recommendations
% belong to charge area, their numbers include the headerstring (eg "FW")
% "\recommendationnum" -> "FW:1"
\newcommand{\recommendationnum}{%
    \headerstring:\therecommendation
}

% \recommendationstring is how the recommendation title will appear in the text.
% "\recommendation[no-windows]{Don't Support Windows}"
%     -> "Recommendation ENV:1 - Don't Support Windows"
% The hyperlink takes you to the list of recommendations, so you can see it in context
\newcommand{\recommendationstring}[2]{\hyperref[tor:\recommendationnum]{\textbf{Recommendation \recommendationnum: We should #2}}}
% Version without hyperlink to list of recommendations:
% \newcommand{\recommendationstring}[2]{\textcolor{DESCred}{\textbf{Recommendation \recommendationnum -- #2}}}

% \recommendationliststring is how the recommendation title will appear in
% the list of recommendations.
%     -> "ENV:1 – Don't Support Windows"
% The hyperlink takes you to the start of the recommendation text.
\newcommand{\recommendationliststring}[1]{\hyperref[\thisrecommendationname]{\textbf{\recommendationnum}} -- we should #1}

% \recommendationrefstring is the text that will appear when a recommendation is cited.
% "recommendation~\recommendationref{no-windows}"  -> "recommendation ENV:1"
\newcommand{\recommendationrefstring}{\recommendationnum}

% \printrecommendation is used to write the recommendation title line itself.
\newcommand{\printrecommendation}[1]{\noindent{#1}}

% \recommendationref is the function to use when referring to a recommendation.
\newcommand{\recommendationref}[1]{\noindent \ref{#1}}

% Which section's table are we goingto be adding to?
\newcommand{\recommendationtable}{\headerstring.tor}

% \makerecommendation is run when \recommendation{} is called. Note the
% use of the paragraph counter to provide an anchor for hyperref.

\newcommand{\makerecommendation}[3]{
   \stepcounter{recommendation}
   \vskip 0.2in
   \refstepcounter{paragraph} % This puts an anchor right before the title line.
   % Write the recommendation title and summary:
   \printrecommendation{\recommendationstring{#1}{#2}}
   \newline\noindent{{\it {#3}}}
   \vskip 0.2in
   % Add line to list of all recommendations, including a label to point to:
   % \addcontentsline{tor}{recommendation}{\small\printrecommendation{\recommendationliststring{#2}}\hfill}
   \addcontentsline{tor}{recommendation}{\leavevmode\newline\small\printrecommendation{\recommendationliststring{#2}}\hfill\protect\phantomsection\protect\label{tor:\recommendationnum}}
   % \myaddcontentsline{tor}{recommendation}{\small\printrecommendation{~~~~\recommendationliststring{#2}}\hfill}{#1}
   % Add line to this sections's table of recommenandations:
   \addcontentsline{\recommendationtable}{recommendation}{\leavevmode\newline\small\printrecommendation{\recommendationliststring{#2}}\hfill}
}

% The recommendation command itself - make a new recommendation (which prints
% a title line, advances the paragraph counter, then prints the summary) and label it for
% referring to it elsewhere.
\newcommand{\recommendation}[3][]{
   \def\thisrecommendationname{#1}
   \def\thisrecommendationtitle{#2}
   \def\thisrecommendationsummary{#2}
   \makerecommendation{#1}{#2}{#3}
   \relabel{\recommendationrefstring}{#1}{paragraph.\thechapter.\theparagraph}
 }


% --------------------------------------------------------------------

% NAVIGATION

\newcommand{\footernavigationbar}
{
    \it\footnotesize Go to:
    \hyperref[toc]{The Table Of Contents} % Table of Contents
    \ifthenelse{\equal{\headerstring}{}}
   	    {}
        {
        	% $\bullet$ \nameref{sec:\headerstring}
            $\bullet$ \hyperref[sec:\headerstring]{the start of this section}
  			$\bullet$ \hyperref[sec:TOR]{the list~of~recommendations}
        }
}

\newcommand{\quickref}[1]{\ref{#1}~\nameref{#1}}

% --------------------------------------------------------------------



%%%%%%%%%%%%%%%%%%%%%%%%%%%%%%%%%%%%%%%%%%%%%%%%%%%%%%%%%%%%%%%%%%%%%%

% Other useful dates:

% US financial years. The first quarter, Q1,
% is taken to be Sept, Oct and Nov. The end of 2017 Q1 would be 11/17.
% Starts of Q1,Q2,Q3,Q4 -> 09/Y-, 12/Y-, 03/YY, 06/YY
%   Ends of Q1,Q2,Q3,Q4 -> 11/Y-, 02/Y-, 05/YY, 08/YY
% This macro works correctly in the text, but cannot be used in place of a
% datestring argument (#1) in the \recommendation command. I (PJM) don't
% know why.
\newcommand{\FY}[2]{%
% eg \FY{16}{1} -> "11/15"
\newcount\year
\ifthenelse{\equal{#2}{Q1}}%
{%
  \year=\numexpr#1-1\relax
  {11/\the\year}%
}{%
  \year=#1%
  \ifthenelse{\equal{#2}{Q2}}{02/\the\year}{%
  \ifthenelse{\equal{#2}{Q3}}{05/\the\year}{%
  {08/\the\year}%
}}}%
}



\newcommand{\deadline}[1]{%
% Parse the datestring provided. Possible options are '??/??',
% '05/17', '5/17', DC1, etc. Default is that date string is
% passed through as is. '??/??' needs
% processing, though.
%
% Relative dates, passed in by keywords like "DC1", will appear in
% regular (not bold) type, to show that they might change with the
% collaboration's schedule.
%
% "DC1" is interpreted to mean the end of the data analysis of DC1.
% Later on we might think about allowing more complex arguments, such as
% "DC1-analysis-start".
%
% US financial years are assumed, so that the first quarter, Q1,
% is taken to be Sept, Oct and Nov. The end of 2017 Q1 would be 11/17.
% Q1,Q2,Q3,Q4 -> 11, 2, 5, 8
%
% First set the correct era.
% This is the big one: any undefined date gets set to the end of
% the current key project's data challenge era:
\ifthenelse{\equal{#1}{??/??}}%
    {\def\era{\thisrecommendationsera}}%
    {\def\era{#1}}%
% This works correctly: era is now either set to "DC1", "DC2" or "DC3"
% (if "??/??" was chosen), or the argument provided.
%
% Inputs to the \recommendation date argument can themselves be macros,
% for example:
%   \recommendation[D:SL-DC1-TDC2:DP1]{\EndofDCOneDC}{\SLTDCtwo Simulated Light Curves}
% This is the preferred way of passing in this information.
% You can also pass in one of the DC names, and it will be
% interpreted as the end date of that DC era, as set by DESC mgmt above.
% This will get set in lightweight font, because its a pretty
% vague timeline!
\ifthenelse{\equal{\era}{DC1}}{\def\duedate{\EndofDCOne}}{%
\ifthenelse{\equal{\era}{DC2}}{\def\duedate{\EndofDCTwo}}{%
\ifthenelse{\equal{\era}{DC3}}{\def\duedate{\EndofDCThree}}{%
% Now we get to the situation where the date was either
% explicitly set, or left as ??/??. The question marks should
% be replaced with \thisrecommendationsera, but if this has happened,
% we set the date in lightweight font. An explicit due date setting is
% set in bold.
\ifthenelse{\equal{#1}{??/??}}{\def\duedate{\era}}{%
% else: the date was set by hand.
\def\duedate{{\bf{\era}}}%
% The following line closes this huge nested if block. There must
% be as many '}' characters on this line as there are 'ifthenelse'
% lines above.
}}}}
% Finally, render the due date, as a hyperlink to the timeline chart:
\hyperref[fig-DC]{\duedate}
}% End of \deadline macro.


% Possible extensions to \deadline
% --------------------------------
%
% Automatically detect the recommendation type from a name like
% "D:KPNAME:TYPE:NUMBER" - which would need all recommendation names to
% follow this convention, probably...
% Splitting strings:
%   http://tex.stackexchange.com/questions/12810/how-do-i-split-a-string
%
% Parse deadlines like DC1-analysis-end?
%
%   Need to be able to split a string like "05/17" and subtract 7 months
%   from it.
%
% Enabling dates minus some number of months?
%
%   See the following for arithmetic in latex:
%   http://tex.stackexchange.com/questions/133317/how-to-add-subtract-multiply-and-divide-in-plain-tex
%
% --------------------------------------------------------------------

% Bibliography stuff:

\bibpunct[, ]{(}{)}{;}{a}{}{,}

% Rename bibliography sections and make everything more compact:
\renewcommand\bibname{\normalsize References}

% PJM (for the Science Book): This command makes latex set up the bibliography as a section*, not
% a chapter. Need to add a markboth command too, below...
% PJM (for the SRM): Not sure this is working...
% \renewcommand\bibsection{\chapter*{\refname}}
\renewcommand\bibsection{\section*{\refname}}
% PJM 10/10/09: need this switch to change marking of appendix headers:
\newboolean{appendix}
\setboolean{appendix}{false}

\let\oldthebibliography=\thebibliography
\let\endoldthebibliography=\endthebibliography
\renewenvironment{thebibliography}[1]{%
  \begin{oldthebibliography}{#1}%
    % Make sure the references line gets into the contents table - note spacing!
    % \addcontentsline{toc}{chapter}{\hspace{2em}References}
    \addcontentsline{toc}{section}{References}
    % \footnotesize
    % \setlength{\parskip}{0ex}%
    % \setlength{\itemsep}{0ex}%
    % PJM 10/10/09: Appendix reference section headers need to be marked with
    % "Appendix" not "Chapter"
    % \ifthenelse{\boolean{appendix}}{
    %   \markboth{Appendix\ \thechapter: References}{Appendix\ \thechapter: References}
    % }{
    %   \markboth{Chapter\ \thechapter: References}{Chapter\ \thechapter: References}
    % }
}%
{%
  \end{oldthebibliography}%
}

%=====================================================================

\makeatother

%=====================================================================
% Add your macros here for all to use!

% LSST Project terminology:
\newcommand{\LevelTwo}{{Level~2}\xspace}
\newcommand{\LevelThree}{{Level~3}\xspace}
\newcommand{\CC}{LSSTComCam\xspace}
\newcommand{\MultiFit}{{\sc MultiFit}\xspace}
\newcommand{\photoz}{photo-$z$\xspace}
\newcommand{\Nbody}{$N$-body\xspace}


% DESC-related software
\newcommand{\astropy}{{\sc astropy}\xspace}
\newcommand{\CAMB}{{\sc CAMB}\xspace}


% Image simulation tools:
\newcommand{\PhoSim}{{\sc PhoSim}\xspace}
\newcommand{\OpSim}{{\sc OpSim}\xspace}
\newcommand{\CatSim}{{\sc CatSim}\xspace}
\newcommand{\GalSim}{{\sc GalSim}\xspace}

% Data Products used by more than one WG
\newcommand{\DADConeHalocat}{\hyperref[KP:DA-DC1-HC]{{\bf\texttt{HaloCat}}}\xspace}%{{\bf\texttt{HaloCat\,1.0}}}\xspace}
\newcommand{\DATwinklesone}{\hyperref[KP:DA-DC1-TW1]{{\bf\texttt{Twinkles\,1}}}\xspace}%{{\bf\texttt{Twinkles\,1.0}}}\xspace}
\newcommand{\DATwinklestwo}{\hyperref[KP:DA-DC2-TW2]{{\bf\texttt{Twinkles\,2}}}\xspace}%{{\bf\texttt{Twinkles\,2.0}}}\xspace}
\newcommand{\DADCtwomockLC}{\hyperref[KP:DA-DC2-LC]{{\bf\texttt{DC2 Mock Lightcone}}}\xspace}
\newcommand{\DADCthreemockCC}{\hyperref[KP:DA-DC3-CC]{{\bf\texttt{DC3 Mock ComCam Survey}}}\xspace}
\newcommand{\DADCthreemockLC}{\hyperref[KP:DA-DC3-LC]{{\bf\texttt{DC3 Mock Lightcone}}}\xspace}
\newcommand{\DADConeDeep}{\hyperref[KP:DA-DC1-PHO-D]{{\bf\texttt{DC1 Phosim Deep}}}\xspace}
\newcommand{\DADConeWide}{\hyperref[KP:DA-DC1-PHO-W]{{\bf\texttt{DC1 Phosim Wide}}}\xspace}
\newcommand{\DADCtwoWide}{\hyperref[KP:DA-DC2-PHO-W]{{\bf\texttt{DC2 Phosim Wide}}}\xspace}
%\newcommand{\DADCtwomocks}{\hyperref[KP:DA-DC2-mock]{{\bf\texttt{DESC DC2 Mock Galaxy Catalogs}}}\xspace}
%\newcommand{\DAGREAT}{\hyperref[KP:DA-DC1-GREAT]{{\bf\texttt{GREAT DESC}}}\xspace}
% Weak Lensing software tools:
\newcommand{\WLImSim}{\hyperref[tab:WL:software]{\sc WLImSim}\xspace}
\newcommand{\WLShearPipe}{\hyperref[tab:WL:software]{\sc WLShearPipe}\xspace}
\newcommand{\WLTwopoint}{\hyperref[tab:WL:software]{\sc WLTwopoint}\xspace}
\newcommand{\WLNullTest}{\hyperref[tab:WL:software]{\sc WLNullTest}\xspace}
\newcommand{\WLCosmo}{\hyperref[tab:WL:software]{\sc WLCosmo}\xspace}
\newcommand{\WLMagPipe}{\hyperref[tab:WL:software]{\sc WLMagPipe}\xspace}
\newcommand{\WLNPoint}{\hyperref[tab:WL:software]{\sc WLNPoint}\xspace}
\newcommand{\WLBlendBias}{\hyperref[tab:WL:software]{\sc WLBlendBias}\xspace}
% LSS definition
\newcommand{\LSScatalog}{\hyperref[tab:LSS:software]{\sc LSScatalog}\xspace}
\newcommand{\LSStwopoint}{\hyperref[tab:LSS:software]{\sc LSStwopoint}\xspace}
\newcommand{\LSSthreepoint}{\hyperref[tab:LSS:software]{\sc LSSthreepoint}\xspace}
\newcommand{\LSSsyspre}{\hyperref[tab:LSS:software]{\sc LSSsyspre}\xspace}
\newcommand{\LSSsyshunter}{\hyperref[tab:LSS:software]{\sc LSSsyshunter}\xspace}
\newcommand{\LSSCross}{\hyperref[tab:LSS:software]{\sc LSSCross}\xspace}
\newcommand{\LSSCosmo}{\hyperref[tab:LSS:software]{\sc LSSCosmo}\xspace}

% Galaxy Clusters software tools
\newcommand{\CLOptCat}{\hyperref[tab:CL:software]{\sc CLOptCat}\xspace}
\newcommand{\CLFinder}{\hyperref[tab:CL:software]{\sc CLFinder}\xspace}
\newcommand{\CLPanCat}{\hyperref[tab:CL:software]{\sc CLPanCat}\xspace}
\newcommand{\CLRedshift}{\hyperref[tab:CL:software]{\sc CLRedshift}\xspace}
\newcommand{\CLSmurfs}{\hyperref[tab:CL:software]{\sc CLSmurfs}\xspace}
\newcommand{\CLMassMod}{\hyperref[tab:CL:software]{\sc CLMassMod}\xspace}
\newcommand{\CLShear}{\hyperref[tab:CL:software]{\sc CLShear}\xspace}
\newcommand{\CLAbsMass}{\hyperref[tab:CL:software]{\sc CLAbsMass}\xspace}
\newcommand{\CLRelMass}{\hyperref[tab:CL:software]{\sc CLRelMass}\xspace}
\newcommand{\CLForecast}{\hyperref[tab:CL:software]{\sc CLForecast}\xspace}
\newcommand{\CLCosmo}{\hyperref[tab:CL:software]{\sc CLCosmo}\xspace}

% Supernova software tools
%   [we can't let SL be the only ones with cool-looking names]
\newcommand{\SNRealizer}{\hyperref[tab:SN:software]{\sc SupernovaRealizer}\xspace}
\newcommand{\SNMonitor}{\hyperref[tab:SN:software]{\sc SupernovaMonitor}\xspace}
\newcommand{\SNType}{\hyperref[tab:SN:software]{\sc SupernovaType}\xspace}
\newcommand{\SNDistance}{\hyperref[tab:SN:software]{\sc SupernovaDistance}\xspace}
\newcommand{\SNCadence}{\hyperref[tab:SN:software]{\sc SupernovaCadence}\xspace}

% Photo-z software tools
\newcommand{\PZGalaxyGenerator}{\hyperref[tab:PZ:software]{\sc PZGalaxyGenerator}\xspace}
\newcommand{\PZPDF}{\hyperref[tab:PZ:software]{\sc PZPDF}\xspace}
\newcommand{\PZCalibrate}{\hyperref[tab:PZ:software]{\sc PZCalibrate}\xspace}
\newcommand{\PZSpeczSelector}{\hyperref[tab:PZ:software]{\sc PZSpeczSelector}\xspace}
%Photo-z Datasets
\newcommand{\PZColor}{\hyperref[KP:PZ:DC12]{\bf\texttt{PZColor}}\xspace}
\newcommand{\PZIncomplete}{\hyperref[KP:PZ:DC12]{\bf\texttt{PZIncomplete}}\xspace}
\newcommand{\PZCrossCorr}{\hyperref[KP:PZ:DC2_cor]{\bf\texttt{PZCrossCorr}}\xspace}
\newcommand{\PZSzTrain}{\hyperref[KP:PZ:ext_spec]{\bf\texttt{PZSzTrain}}\xspace}
\newcommand{\PZNIR}{\hyperref[KP:TJP-DC3-kp3]{\bf\texttt{PZNIR}}\xspace}

% Strong Lensing software tools:
\newcommand{\SLFinder}{\hyperref[tab:SL:software]{\sc SLFinder}\xspace}
\newcommand{\SLMonitor}{\hyperref[tab:SL:software]{\sc SLMonitor}\xspace}
\newcommand{\SLTimer}{\hyperref[tab:SL:software]{\sc SLTimer}\xspace}
\newcommand{\SLMassMapper}{\hyperref[tab:SL:software]{\sc SLMassMapper}\xspace}
\newcommand{\SLEnvCounter}{\hyperref[tab:SL:software]{\sc SLEnvCounter}\xspace}
\newcommand{\SLModeler}{\hyperref[tab:SL:software]{\sc SLModeler}\xspace}
\newcommand{\SLCosmo}{\hyperref[tab:SL:software]{\sc SLCosmo}\xspace}
% Strong Lensing datasets:
\newcommand{\SLTDCtwo}{\hyperref[KP:SL-DC1-TDC2]{\texttt{TDC2}}\xspace}
\newcommand{\SLTDCthree}{\hyperref[KP:SL-DC3-TDC3]{\texttt{TDC3}}\xspace}
\newcommand{\SLDCtwoReal}{\hyperref[]{\texttt{DC2 Real Data}}\xspace}
% Theory / Joint Probes software tools
\newcommand{\TJPForecast}{\hyperref[tab:TJP:software]{\sc TJPForecast}\xspace} % Joint probe covariance matrix
\newcommand{\TJPCosmo}{\hyperref[tab:TJP:software]{\sc TJPCosmo}\xspace} % Joint probe covariance matrix
\newcommand{\TJPCov}{\hyperref[tab:TJP:software]{\sc TJPCov}\xspace} % Joint probe covariance matrix

% Cosmology analysis tools
\newcommand{\CosmoSIS}{{\sc CosmoSIS}\xspace}
\newcommand{\CosmoLike}{{\sc CosmoLike}\xspace}

% Journals
\newcommand{\apj}{ApJ}
\newcommand{\apjl}{ApJL}
\newcommand{\apjs}{ApJS}
\newcommand{\mnras}{MNRAS}
\newcommand{\apss}{Ap \& SS}
\newcommand{\aap}{A\&A}
\newcommand{\aaprlong}{Astronomy and Astrophysics Reviews}
\newcommand{\aj}{AJ}
\newcommand{\prd}{Phys. Rev. D}
\newcommand{\nat}{Nature}
\newcommand{\araa}{ARA\&A}
\newcommand{\jgr}{J. Geophys. Res.}
\newcommand{\pasp}{PASP}
\newcommand{\aapr}{AAPR}
\newcommand{\jcap}{JCAP}
\newcommand{\pasj}{PASJ}

% Bits and pieces
\newcommand{\etal}{et~al.~}
\newcommand{\eg}{{\it e.g.\ }}
\newcommand{\ie}{{\it i.e.\ }}
\newcommand{\etc}{{\it etc.\ }}
\newcommand{\dndmag}{d$N$/dmag}
\newcommand{\neff}{\ensuremath{n_\text{eff}}}

% Colors, commenting etc:
\definecolor{gray}{rgb}{0.5,0.5,0.5}
\definecolor{green}{rgb}{0.0,0.5,0.0}
\definecolor{DESCred}{rgb}{0.63,0.00,0.20} % {#A10034}
\definecolor{dkblue}{rgb}{0, 0.30, 0.99 }%{0,76,253}
% Guidance at http://latexcolor.com/
% \newcommand{\question}[2]{\textcolor{red}{\bf Question from #1: #2}}
\newcommand{\flag}[2]{\textcolor{red}{\bf #1: #2}}
% \newcommand{\new}[1]{\textcolor{green}{#1}}
\newcommand{\black}[1]{\textcolor{black}{#1}}
\newcommand{\gray}[1]{\textcolor{gray}{#1}}

%\newcommand{\new}[1]{\textcolor{blue}{#1}}
%\newcommand{\question}[2]{\textcolor{blue}{{\bf Question from #1: #2}}} % eg \question{PJM}{Shouldn't this be ``better''?}
%\newcommand{\comment}[2]{\textcolor{blue}{{\bf #1: #2}}} % eg \comment{RXD}{This will likely be too expensive.}
\newcommand{\hlt}[1]{\textcolor{blue}{{#1}}}

% Cross-referencing definitions:
\newcommand{\define}[1]{\textcolor{DESCred}{\it{\bf #1}}\phantomsection\label{#1}}
\newcommand{\lookup}[1]{\hyperref[#1]{#1}}

%=====================================================================

\newcounter{requirementh}
\newcounter{requirementd}
\newcounter{objectives}
\newcounter{goal}
\newcommand{\probe}{\@empty}

\newcommand{\requirementhname}{%
    RH{\therequirementh}%
}
\newcommand{\requirementdname}{%
    \probe{\therequirementd}%
}
\newcommand{\objectivename}{%
    O{\theobjectives}%
}
\newcommand{\goalname}{%
    G{\thegoal}%
}

% Custom labeling; requires hyperref!
\makeatletter
\newcommand{\customlabel}[2]{%
   \protected@write \@auxout {}{\string \newlabel {#1}{{#2}{\thepage}{#2}{#1}{}} }%
   \hypertarget{#1}{\textbf{#2}:}
}
\makeatother

\makeatletter
\newcommand{\labelreqh}[2]{%
   \protected@write \@auxout {}{\string \newlabel {#1}{{\requirementhname}{\thepage}{\requirementhname}{#1}{}} }%
   \hypertarget{#1}{\textbf{\textcolor{cyan}{High-level requirement \requirementhname}: #2}}\stepcounter{requirementh}}
\makeatother

\makeatletter
\newcommand{\labelreqdjoint}[3]{%
   \protected@write \@auxout {}{\string \newlabel {#1}{{\requirementdname}{\thepage}{\requirementdname}{#1}{}} }%
   \hypertarget{#1}{\textbf{\textcolor{cyan}{Detailed joint probes requirement {\requirementdname} {#2}}: {#3}}}\stepcounter{requirementd}}
\makeatother

\makeatletter
\newcommand{\labelreqd}[4]{%
   \protected@write \@auxout {}{\string \newlabel {#1}{{\requirementdname}{\thepage}{\requirementdname}{#1}{}} }%
   \hypertarget{#1}{\textbf{\textcolor{cyan}{Detailed requirement {\requirementdname} (Y10)}: Systematic uncertainty in {#2} shall not exceed {#4} in the Y10 DESC {\probe} analysis.\\ \textcolor{cyan}{Goal {\requirementdname} (Y1)}: Systematic uncertainty in {#2} should not exceed {#3} in the Y1 DESC {\probe} analysis.}}\stepcounter{requirementd}}
\makeatother

\makeatletter
\newcommand{\labelreqdcomment}[5]{%
   \protected@write \@auxout {}{\string \newlabel {#1}{{\requirementdname}{\thepage}{\requirementdname}{#1}{}} }%
   \hypertarget{#1}{\textbf{\textcolor{cyan}{Detailed requirement {\requirementdname} (Y10)}: Systematic uncertainty in {#2} shall not exceed {#4} in the Y10 DESC {\probe} analysis. #5\\ \textcolor{cyan}{Goal {\requirementdname} (Y1)}: Systematic uncertainty in {#2} should not exceed {#3} in the Y1 DESC {\probe} analysis.}}\stepcounter{requirementd}}
\makeatother

\makeatletter
\newcommand{\labelobj}[2]{%
   \protected@write \@auxout {}{\string \newlabel {#1}{{\objectivename}{\thepage}{\objectivename}{#1}{}} }%
   \hypertarget{#1}{\textbf{\textcolor{cyan}{Objective \objectivename}: #2}}\stepcounter{objectives}}
\makeatother

\makeatletter
\newcommand{\labelg}[2]{%
   \protected@write \@auxout {}{\string \newlabel {#1}{{\goalname}{\thepage}{\goalname}{#1}{}} }%
   \hypertarget{#1}{\textbf{\textcolor{cyan}{Goal \goalname}: #2}}\stepcounter{goal}}
\makeatother
