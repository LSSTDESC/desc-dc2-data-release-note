\begin{ThreePartTable}
\begin{TableNotes}
\end{TableNotes}
\begin{longtable}{p{1.6in}p{0.5in}p{0.6in}p{2.9in}}
\hline
\textbf{Name} & \textbf{Type} & \textbf{Unit} & \textbf{Description} \\ 
\hline
\endhead
%%%\endfoot
%%%\hline
%%%%\insertTableNotes  % tell LaTeX where to insert the contents of "TableNotes"
%%%\endlastfoot
\code{id} & int64 & -- & Unique object ID\\
\code{ra} & float64 & degree & Right Ascension\\
\code{dec} & float64 & degree & Declination\\
\code{flux_<band>} & float32 & nJy & Static flux value in <band>\\
\code{model} & string & -- & variability model\\
\code{max_stdev_delta_mag} & float32 & AB mag & max stdev in delta magnitude across all bands\\
\code{above_threshold} & bool & -- & True if above quantity satisfies threshold (here set to 1 mmag) for LSST detectability\\
\code{av} & float32 & -- & Milky Way total extinction parameter $A(V)$ at object location \\ 
\code{rv} & float32 & -- &  Milky Way relative visibility parameter $R(V) = A(V)/E(B-V)$ at object location \\
\hline
\end{longtable}
\end{ThreePartTable}