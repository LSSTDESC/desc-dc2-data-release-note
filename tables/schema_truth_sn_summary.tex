\begin{ThreePartTable}
%%%\begin{TableNotes}
%%%\end{TableNotes}
\begin{longtable}{p{1.6in}p{0.5in}p{0.6in}p{2.9in}}
\hline
\textbf{Name} & \textbf{Type} & \textbf{Unit} & \textbf{Description} \\ 
\hline
\endhead
%%%\endfoot
%%%\hline
%%%\insertTableNotes  % tell LaTeX where to insert the contents of "TableNotes"
%%%\endlastfoot
\code{id_string} & string & -- & Unique object ID\\
\code{id} & int64 & -- & alternate id which is an int\\
\code{host_galaxy} & int64 & -- & associated galaxy\\
\code{ra} & float64 & degree & Right Ascension\\
\code{dec} & float64 & degree & Declination\\
\code{redshift} & float64 & -- & Redshift\\
\code{mB} & float64 & -- & normalization factor \\
\code{c} & float64 & -- & empirical parameter controlling the colors\\
\code{t0} & float64 & -- & time in mjd of phase=0, corresponding to B-band maximum \\
\code{x0} & float64 & -- & normalization factor\\
\code{x1} & float64 & -- & empirical parameter controlling the stretch in time of light curves\\
\code{max_flux_<band>} & float32 & nJy & max flux observed for <band>\\
\code{av} & float32 & --  & Milky Way extinction parameter at object location\\
\code{rv} & float32 & -- & Milky Way extinction parameter at object location\\
\end{longtable}
\end{ThreePartTable}