% ====================================================================
%
% LSST DESC DC2 Data Release Note
%
% ====================================================================

\documentclass[11pt]{report}

\include{inc/doc_settings} % formatting for doc, copied from SRM

\include{inc/macros}       % commands and abbreviations, adapted from SRM doc_macros.tex

% ====================================================================
\usepackage{float,longtable,threeparttablex,booktabs}

\begin{document}
\pagenumbering{gobble}

%%%%% TITLE PAGE
\pagestyle{empty}

\vspace*{0.2\textheight}

\begin{center}
{\Huge\bfseries The LSST Dark Energy Science Collaboration (DESC)\\
\bigskip DC2 Data Release Note}

\vspace*{0.2\textheight}

{\Large\bfseries Version~1}
% {\Large\bfseries Version~0.99.\href{COMMITURL}{\texttt{SHA}}} \\
% This is a current, intermediate version of the LSST DESC SRD \\
% which contains edits relative to
% % the formally released \href{\urlvonepointzero}{version v1.0}.
% \href{https://github.com/LSSTDESC/Requirements/releases/tag/v0.9}{version v0.9}.

\medskip
% Branch: \href{BRANCHURL}{\texttt{BRANCH}} \\
% Most recent commit: \href{COMMITURL}{\texttt{SHA}} \\
Date: December 2, 2020


\vspace*{0.1\textheight}

\begin{figure}[!h]
\centering\includegraphics[width=5cm,angle=0]{inc/desc-logo.png}
\end{figure}

\end{center}

% %%%%% VERSION HISTORY
\clearpage
\newpage
\section*{Change Record}

\begin{table}[!thp]%[htdp]
\begin{center}

\renewcommand{\arraystretch}{1.2}
\begin{tabular}{|p{1.5cm}|p{2cm}|p{6.cm}|p{5cm}|}
\hline
{\bf Version}  &{\bf Date}  &{\bf Description}  & {\bf Owner name}
\\ \hline
  \href{}{}
  &
  &
  &
\\ \hline
  \href{}{}
  &
  &
  &
\\ \hline
\end{tabular}
\renewcommand{\arraystretch}{1.0}
\end{center}
\end{table}


\clearpage

\phantomsection\section*{Contributors}
% Force the Contributor listing to show up in the TOC even though it's a section*:
\addcontentsline{toc}{section}{Contributors}

\noindent Contributors to the DESC DC2 Data Release effort are listed in the table below.
%, with leading contributions shown in bold and affiliations indicated in the notes below the table.

\begin{ThreePartTable}
\begin{TableNotes}
\footnotesize
\item [1] Your school
\end{TableNotes}
\begin{longtable}{|p{4cm}|p{11cm}|}
\endfirsthead
\endhead
\endfoot
\insertTableNotes  % tell LaTeX where to insert the contents of "TableNotes"
\endlastfoot
\hline
Name & Contribution \\ \hline
Your name$^{1}$ & Your contribution \\
\hline
\end{longtable}
\end{ThreePartTable}



% %%%%% TABLE OF CONTENTS
\clearpage
{\let\cleardoublepage\clearpage}

\vspace*{-1.0in}
\maketoc
\label{toc}

\clearpage

%%%%%%%% FOOTER
\pagestyle{fancy}
\fancyfoot{} % clear all footer fields
\fancyfoot[R]{\thepage}  % page number in "outer" position of footer line
% \fancyfoot[L]{\footernavigationbar}  % navigation to contents etc

%%%%%%%% HEADER
\fancyhead[L]{}
\fancyhead[R]{LSST DESC DC2 Data Release}
\renewcommand{\footrulewidth}{1pt}

\setlength\parindent{0em}
\setlength{\parskip}{0.5em}
% ====================================================================

% % FRONT MATTER
% \renewcommand{\thepage}{\roman{page}}% Roman numerals for page counter
% \setcounter{page}{0}

% ====================================================================

\renewcommand{\thepage}{\arabic{page}}
\setcounter{page}{1}

\renewcommand\thefigure{\arabic{figure}}
\setcounter{figure}{0}

% ====================================================================

\phantomsection\section*{Executive Summary and Quick Start Guide}
% Force the Executive Summary to show up in the TOC even though it's a section*:
\addcontentsline{toc}{section}{Executive Summary and Quick Start Guide}




\section{Introduction}

{\bf UNDER CONSTRUCTION!}

\begin{itemize}
\item Description of the purpose of the LSST DESC data challenges, including DC1 and DC2:
\item Very brief overview of the generation
\item Some example use cases
\end{itemize}

In the next decade the Vera C. Rubin Observatory will carry out an unprecedented survey of the sky, the Legacy Survey of Space and Time \citep{2009arXiv0912.0201L}. One of the major aims of the survey is to unravel the origin of the cause for the accelerated expansion of the Universe. The LSST Dark Energy Science Collaboration (DESC) was formed to carry out this exciting endeavor~\citep{Abate:2012za}. In order to prepare for the arrival of data, LSST DESC has prepared two data challenges, DC1 and DC2, based on sophisticated image simulations. The data challenges have been design to mimic actual data from the Rubin Observatory as closely as possible. 

GENERAL DESCRIPTION OF THE DATA CHALLENGES: Both data challenges are based on realistic simulations of the extragalactic sky, imSim, LSST Science Pipelines~\citep{2017ASPC..512..279J} TBC. The first data challenge, DC1, covers a $\sim$40 deg$^2$ area and ten years of observations. The image simulations were carried out for $r$-band only. A detailed description and a range of analysis results are provided in~\cite{dc1}. 

In this data release note, we focus on the second data challenge, DC2. A comprehensive description of the LSST DESC DC2 Simulated Sky Survey can be found in~\cite{2020arXiv201005926L}. For completeness, we provide a brief description here. MORE TEXT HERE. The underlying extragalactic catalog, cosmoDC2, is described in~\cite{korytov} and is publicly available\footnote{here...}.



This note is organized as follows. In~\autoref{sec:features} we describe the major features of the DC2 dataset.
We provide an overview of the data products that are part of this release in Section~\autoref{sec:products}.
We provide instructions for how to access the data in~\autoref{sec:access}.
Together with the data itself, we release a set of selected notebooks that contain examples for common data analysis tasks.
The notebooks are introduced in~\autoref{sec:notebooks}.
%In Section~\autoref{sec:validation} we present our validation and characterization of the DC2 data products.
We conclude in~\ref{sec:outlook} and provide a brief description of possible future data releases. 

\section{Major Features of the Dataset}
\label{sec:features}

\section{Data Products}
\label{sec:products}

\subsection{Object Catalog}

\begin{ThreePartTable}
\begin{TableNotes}
\footnotesize
\item [\hypertarget{obj_fn1}{1}] In LSE-163, \code{I<xx,yy,xy>} and \code{I<xx,yy,xy>PSF} are defined in the units of squared arcsec. 
\end{TableNotes}
\begin{longtable}{p{1.7in}p{0.5in}p{0.6in}p{2.8in}}
\hline
\textbf{Name} & \textbf{Type} & \textbf{Unit} & \textbf{Description} \\ 
\hline
\endhead
\endfoot
\hline
\insertTableNotes  % tell LaTeX where to insert the contents of "TableNotes"
\endlastfoot
\code{objectId} & int64 & -- & Unique object ID \\
\code{parentObjectId} & int64 & -- & Parent object ID \\
%
\code{good} & bool & -- & \code{True} if the source has no flagged pixels \\
\code{clean} & bool & -- &  \code{True} if the source has no flagged pixels (i.e., \code{good}) and is not skipped by the deblender \\
\code{blendedness} & float64 & -- & Measure of how flux is affected by neighbors ($1 - I_\text{child}/I_\text{parent}$; see \S\,4.9.11 of \citealt{10.1093/pasj/psx080}) \\
\code{extendedness} & float64 & -- & 0 for stars; 1 for extended objects \\
\code{ra} & float64 & degree & Right Ascension \\
\code{dec} & float64 & degree & Declination \\
\code{x} & float64 & pixel & 2D centroid location (x coordinate) \\
\code{y} & float64 & pixel & 2D centroid location (y coordinate) \\
\code{xErr} & float32 & pixel & Error value for \code{x} \\
\code{yErr} & float32 & pixel & Error value for \code{y} \\
\code{xy_flag} & bool & -- & Flag for issues with \code{x} and \code{y} \\
\code{tract} & int64 & -- & Tract ID in Sky Map \\ 
\code{patch} & string & -- & Patch ID in Sky Map (as a string, \code{`x,y'})\\ 
%
\code{Ixx_pixel} & float64 & sq.~pixel$^\text{\hyperlink{obj_fn1}{1}}$ & Adaptive second moment ($xx$) of source intensity, averaged across bands \\
\code{Ixx_pixel_<band>} & float64 & sq.~pixel$^\text{\hyperlink{obj_fn1}{1}}$ & Adaptive second moment ($xx$) of source intensity in \code{<band>} \\
\code{Iyy_pixel} & float64 & sq.~pixel$^\text{\hyperlink{obj_fn1}{1}}$ & Adaptive second moment ($yy$) of source intensity, averaged across bands \\
\code{Iyy_pixel_<band>} & float64 & sq.~pixel$^\text{\hyperlink{obj_fn1}{1}}$ & Adaptive second moment ($yy$) of source intensity in \code{<band>} \\
\code{Ixy_pixel} & float64 & sq.~pixel$^\text{\hyperlink{obj_fn1}{1}}$ & Adaptive second moment ($xy$) of source intensity, averaged across bands \\
\code{Ixy_pixel_<band>} & float64 & sq.~pixel$^\text{\hyperlink{obj_fn1}{1}}$ & Adaptive second moment ($xy$) of source intensity in \code{<band>} \\
\code{I_flag} & bool & -- & Flag for issues with \code{Ixx}, \code{Iyy_pixel}, and \code{Ixy} \\
\code{I_flag_<band>} & bool & -- & Flag for issues with \code{Iyy_pixel_<band>}, \code{Ixy_<band>}, and \code{Ixx_<band>} \\
\code{IxxPSF_pixel} & float64 & sq.~pixel$^\text{\hyperlink{obj_fn1}{1}}$ & Adaptive second moment ($xx$) of PSFy, averaged across bands \\
\code{IxxPSF_pixel_<band>} & float64 & sq.~pixel$^\text{\hyperlink{obj_fn1}{1}}$ & Adaptive second moment ($xx$) of PSF in \code{<band>} \\
\code{IyyPSF_pixel} & float64 & sq.~pixel$^\text{\hyperlink{obj_fn1}{1}}$ & Adaptive second moment ($yy$) of PSFy, averaged across bands \\
\code{IyyPSF_pixel_<band>} & float64 & sq.~pixel$^\text{\hyperlink{obj_fn1}{1}}$ & Adaptive second moment ($yy$) of PSF in \code{<band>} \\
\code{IxyPSF_pixel} & float64 & sq.~pixel$^\text{\hyperlink{obj_fn1}{1}}$ & Adaptive second moment ($xy$) of PSFy, averaged across bands \\
\code{IxyPSF_pixel_<band>} & float64 & sq.~pixel$^\text{\hyperlink{obj_fn1}{1}}$ & Adaptive second moment ($xy$) of PSF in \code{<band>} \\
\code{psf_fwhm_<band>} & float64 & arcsec & PSF FWHM calculated from \code{base_SdssShape} \\
\code{psNdata} & float32 & - & Number of data points (pixels)
used to fit the model \\
%
\code{psFlux_<band>} & float64 & nJy & Point-source model flux in \code{<band>} \\
\code{psFluxErr_<band>} & float64 & nJy & Error value for \code{psFlux_<band>} \\
\code{psFlux_flag_<band>} & bool & -- & Flag for issues with \code{psFlux_<band>} \\
\code{mag_<band>} & float64 & AB mag & Point-source model magnitude in \code{<band>}\\
\code{magerr_<band>} & float64 & AB mag & Error value for \code{mag_<band>} \\
%
\code{cModelFlux_<band>} & float64 & nJy & Composite model (cModel) flux in \code{<band>} \\
\code{cModelFluxErr_<band>} & float64 & nJy & Error value for \code{cModelFlux_<band>} \\
\code{cModelFlux_flag_<band>} & bool & -- & Flag for issues with \code{cModelFlux_<band>} \\
\code{mag_<band>_cModel} & float64 & AB mag & cModel magnitude in \code{<band>} \\
\code{magerr_<band>_cModel} & float64 & AB mag & Error value for \code{mag_<band>_cModel} \\
\code{snr_<band>_cModel} & float64 & -- & Signal-to-noise ratio for cModel magnitude in \code{<band>} \\
\end{longtable}
\end{ThreePartTable}

\subsection{Truth-match Catalog}

\begin{ThreePartTable}
\begin{TableNotes}
\footnotesize
\item [\hypertarget{truth_fn1}{1}] When accessing this catalog as \code{dc2_object_run2.2i_dr6_wfd_with_truth_match} via \code{GCRCatalogs}, all the columns in this table, except for the last six, are postfixed with \code{_truth}.
\item [\hypertarget{truth_fn2}{2}] Only present when accessing the catalog via \code{GCRCatalogs}.
\item [\hypertarget{truth_fn3}{3}] Because the object catalog is used as the reference catalog for matching, some truth entries may appear more than once, and some truth entries may not have a matching object.
\end{TableNotes}
\begin{longtable}{p{1.6in}p{0.5in}p{0.6in}p{2.9in}}
\hline
\textbf{Name}$^\text{\hyperlink{truth_fn1}{1}}$ & \textbf{Type} & \textbf{Unit} & \textbf{Description} \\ 
\hline
\endhead
\endfoot
\hline
\insertTableNotes  % tell LaTeX where to insert the contents of "TableNotes"
\endlastfoot
\code{id} & string & -- & Unique object ID \\ 
\code{host_galaxy} & int64 & -- & ID of the host galaxy for a SN entry ($-1$ for other truth types) \todo{check with Joanne}\\ 
\code{ra} & float64 & degree & Right Ascension \\
\code{dec} & float64 & degree & Declination \\
\code{redshift} & float32 & -- & Redshift \\ 
\code{is_variable} & int32 & -- & 1 for a variable source \\ 
\code{is_pointsource} & int32 & -- & 1 for a point source \\ 
\code{flux_<band>} & float32 & nJy & Static flux value in \code{<band>} \\ 
\code{flux_<band>_noMW} & float32 & nJy & Static flux value in \code{<band>}, without Milky Way extinction (i.e., dereddened) \\ 
\code{mag_<band>}$^\text{\hyperlink{truth_fn1}{2}}$ & float32 & AB mag & Magnitude in \code{<band>} \\ 
\code{mag_<band>_noMW}$^\text{\hyperlink{truth_fn1}{2}}$ & float32 & AB mag & Magnitude in \code{<band>}, without Milky Way extinction (i.e., dereddened) \\ 
\code{tract} & int64 & -- & Tract ID in Sky Map \\ 
\code{patch} & string & -- & Patch ID in Sky Map (as a string, \code{`x,y'}) \\ 
\code{cosmodc2_hp} & int64 & -- & Healpix ID in cosmoDC2 (for galaxies only; $-1$ for stars and SNe) \\ 
\code{cosmodc2_id} & int64 & -- & Galaxy ID in cosmoDC2 (for galaxies only; $-1$ for stars and SNe)\\ 
\code{truth_type} & int64 & -- & 1 for galaxies, 2 for stars, and 3 for SNe \\ 
\code{match_objectId} & int64 & -- & \code{objectId} of the matching object entry ($-1$ for unmatched truth entries$^\text{\hyperlink{truth_fn3}{3}}$) \\ 
\code{match_sep} & float64 & arcsec & On-sky angular separation of this object--truth matching pair ($-1$ for unmatched truth entries$^\text{\hyperlink{truth_fn3}{3}}$) \\ 
\code{is_good_match} & bool & -- & True if this object--truth matching pair satisfies the matching criteria \\
\code{is_nearest_neighbor} & bool & -- & True if this truth entry is the nearest neighbor of the object specified by \code{match_objectId} \\
\code{is_unique_truth_entry} & bool & -- & True for truth entries that appear for the first time in this truth table$^\text{\hyperlink{truth_fn3}{3}}$ \\
\end{longtable}
\end{ThreePartTable}


\section{Data Access}
\label{sec:access}

\subsection{Obtaining Data Files via Globus Online}

\subsection{Accessing Data Files via GCRCatalogs}

\section{Example Notebooks}
\label{sec:notebooks}

\section{Validation and Characterization}
\label{sec:validation}

\subsection{Tests of the Object Table}

These tests can be asked of any coadd catalog from real or simulated data.

\begin{itemize}
  \item Do we cover the RA and Dec as expected
  \item Are stars point-like and galaxies extended?
  \item Is the distribution of measured PSFs reasonable.  Note that this is on the coadds for the current level of processing, but the coadds should be representative of the distribution of PSFs from the input images.
  \item Does a color-color diagram of stars look reasonable?
  \item Are the image 5-sigma point-source depths reasonable?
  \item Is the areal density of galaxies on the sky reasonable?
      \begin{itemize}
          \item Check both versus other surveys using real data of the real sky. 
      \end{itemize}
  \item Is the number density of galaxies vs. brightness (say i band magnitude) reasonable?
      \begin{itemize}
          \item Check both versus other surveys using real data of the real sky. 
      \end{itemize}
  \item Is the separation of stars vs. galaxies decent.  Does it behave as expected as a function of SNR?  Are there unexpected features?
  \item Is the distribution of shape, ellipticity, and T second moments reasonable and as expected?
  \item Is the distribution of [shear-based] e1, e2, $e_abs$ reasonable?  Are there patterns in the distribution across the sky?
  \item What's the relationship between different types of extended-object photometry?
  \item Do Central Cluster galaxies follow the red sequence?
  \begin{itemize}
      \item Is the cluster red sequence visible in z vs. g-r and r-z vs. g-r plots of extended objects in the data?
      \item If you run a cluster finder, and then identify central galaxies, do you find that the expected central cluster galaxy red sequence pops out clearly?
  \end{itemize}

\end{itemize}

\subsection{Comparison of the Object Table with Truth.}

Comparison against the known characteristics of the models without matching:

\begin{itemize}
  \item Does the areal density of stars match the input model of the Milky Way?
  \item Does the number density of galaxies vs. brightness (say i band magnitude) match the input model?
\end{itemize}

The following questions can only be asked after matching the observered catalog to the truth catalogs.

\begin{itemize}
  \item Observed - Truth magnitude
  \begin{itemize}
      \item For stars, galaxies.  Point source and extended model photometry.
      \item Is the distribution centered at 0?
      \item Is it symmetric or skewed?
      \item Is the skew consistent with (reverse) Malmquist bias at bright end?
      \item As a function of brightness.
        \item Does the scatter at the low SNR end follow the reported uncertainties?
        \item What happens at the bright end? Are the effects of saturation visible?
      \item Is a constant offset?  If so, is it consistent with a bookkeeping error somewhere?
      \item What's the behavior for a SNR > 10 cut.
  \end{itemize}

  \item Galaxies
  \begin{itemize}
      \item Do Central Cluster galaxies follow the red sequence?
        \item Look in r-z vs g-r
        \item This can be done using the truth catalog identify cluster members.  And the presented matched plots
        \item Compare truth \item data vs. redshift
        \item Compare truth -data vs. g-r color.
        \item This analysis complements the data-only analysis above.
    \end{itemize}
\end{itemize}

\section{Conclusion and Outlook}
\label{sec:outlook}


\clearpage

\phantomsection\section*{Acknowledgments}
\addcontentsline{toc}{section}{Acknowledgements}

LSST DESC acknowledges ongoing support from the Institut National de Physique Nucl\'eaire et de Physique des Particules in France; the Science \& Technology Facilities Council in the United Kingdom; and the Department of Energy, the National Science Foundation, and the LSST Corporation in the United States. LSST DESC uses the resources of the IN2P3/CNRS Computing Center (CC-IN2P3--Lyon/Villeurbanne - France) funded by the Centre National de la Recherche Scientifique; the Univ. Savoie Mont Blanc - CNRS/IN2P3 MUST computing center; the National Energy Research Scientific Computing Center, a DOE Office of Science User Facility supported by the Office of Science of the U.S.\ Department of Energy under contract No.\ DE-AC02-05CH11231; STFC DiRAC HPC Facilities, funded by UK BIS National E-infrastructure capital grants; and the UK particle physics grid, supported by the GridPP Collaboration. This research used resources of the Argonne Leadership Computing Facility, which is a DOE Office of Science User Facility supported under Contract DE-AC02-06CH11357. This work was performed in part under DOE contract DE-AC02-76SF00515.


% Individual acknowledgments (sorted by author order)
The work of APH, KH, EK, PL, and ASV at Argonne National Laboratory was supported under the U.S. DOE contract DE-AC02-06CH11357.




% %%%%%%%% END MATTER
\clearpage


\bibliographystyle{inc/apj-mod}
\phantomsection\bibliography{ref}


\clearpage
\appendix
\renewcommand\thefigure{\thesection\arabic{figure}}
\renewcommand\thetable{\thesection\arabic{table}}
\renewcommand{\thesection}{\Alph{section}}
\renewcommand{\thesubsection}{\Alph{section}\arabic{subsection}}

\phantomsection\section*{Appendices}
\addcontentsline{toc}{section}{Appendices}





\end{document}

% ====================================================================
